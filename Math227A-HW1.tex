\documentclass{article}
\usepackage[utf8]{inputenc}

\title{Math227A-HW1}
\author{Zachary DeStefano }
\date{September 30, 2016}

\begin{document}

\maketitle

\section{Problem 1}

\subsection{Part A}

This is a linear ODE
\[
p(t) = 1
\]
\[
q(t) = t e^{-t} + 1
\]

The integrating factor $E(t)$ is as follows: $E(t)=e^t$
This means the solution looks like the following:
\[
y = \frac{1}{e^t} ( \int e^t (t e^{-t} + 1) dt + C)
\]
Simplifying and solving the integral we obtain
\[
y = \frac{1}{e^t} (\int (t + e^t)dt + C)
\]
\[
y = \frac{1}{e^t} (\frac{t^2}{2} + e^t + C)
\]
Finally we obtain
\[
y = \frac{t^2}{2 e^t} + C
\]

\newpage

\subsection{Part B}

We start with
\[
2y' - y = e^{t/3}
\]
Dividing by 2 on both sides
\[
y' - \frac{y}{2} = \frac{e^{t/3}}{2}
\]
We now have a linear ODE
\[
p(t) = -1/2
\]
\[
q(t) = \frac{e^{t/3}}{2}
\]
The integrating factor is thus as follows
\[
E(t) = e^{-t/2}
\]
Thus the solution will be as follows
\[
y = \frac{e^{t/2}}{2} ( \int e^{-t/2}e^{t/3} dt + C)
\]
\[
y = \frac{e^{t/2}}{2} ( \int e^{t/6} dt + C)
\]
Integrating we obtain
\[
y = 3e^{t/2} ( e^{t/6} + C)
\]
Finally we have
\[
y = 3\sqrt[3]{e^{2t}} + C \sqrt{e^{t}}
\]

\newpage

\section{Problem 2}

\subsection{Part A}

Using separation of variables we obtain
\[
\frac{dy}{y} = 5dx
\]
Integrating both sides gives us the following
\[
ln(y) = 5x + C
\]
The final solution is thus
\[
y = A e^{5x}
\]
for some positive constant A

\subsection{Part B}
Our equation is as follows
\[
\frac{dy}{dt} = (t+1)^2 y
\]
Separating the variables we obtain
\[
\frac{dy}{y} = (t+1)^2 dt
\]
Integrating both sides we obtain
\[
ln(y) = \frac{1}{3} (t+1)^3 + C
\]
Solving for y we obtain
\[
y = A \sqrt[3]{e^{(t+1)^3}}
\]
for some positive constant A

\newpage

\subsection{Part C}

Rearranging this equation we obtain
\[
y' = -2t y^2
\]
Separating variables we obtain
\[
\frac{dy}{y^2} = -2t dt
\]
Integrating both sides we obtain
\[
\frac{-1}{y} = -t^2 + C
\]
Our final solution is thus as follows
\[
y = \frac{1}{t^2+C}
\]
for some constant C

\subsection{Part D}

Rearranging and separating variables we obtain
\[
\frac{dy}{\sqrt{1-y^2}} = t dt
\]
Integrating both sides we obtain
\[
arcsin(y) = \frac{t^2}{2} + C
\]
The final solution is thus
\[
y = sin(\frac{1}{2} t^2 + C)
\]
for some constant C. 
\newpage

\section{Problem 3}

\subsection{Part A}

Rearranging we have
\[
\frac{dy}{y} = -2dx
\]
Integrating we have
\[
ln(y) = -2x + C
\]
Thus the general solution is as follows
\[
y = Ae^{-2x}
\]
for some positive constant A\\
Since y(0)=2 we can say that
\[
2 = Ae^{-2*0}
\]
which simplifies to $A=2$\\
Thus our unique solution is as follows
\[
y = 2 e^{-2x}
\]
\newpage
\subsection{Part B}

Rearranging we end up with
\[
\frac{dy}{y^2-1} = \frac{dt}{t^2-1}
\]
After doing partial fractions we have
\[
\frac{1}{2}(\frac{1}{y-1} - \frac{1}{y+1})dy = \frac{1}{2}(\frac{1}{t-1} - \frac{1}{t+1})dt
\]
After multiplying both sides by 2 and integrating both sides we have
\[
ln(1-y) - ln(1+y) = ln(1-t) - ln(1+t) + C
\]
Simplifying we obtain
\[
ln(\frac{1-y}{1+y}) = C + ln(\frac{1-t}{1+t})
\]
Exponentiating both sides we obtain
\[
(1-y)(1+t) = A (1-t)(1+y)
\]
To solve for A, I plug in the initial condition of $y=1$, $t=2$\\
The equation then becomes $A=0$ \\
It must thus hold that either $1-y=0$ or $1+t=0$\\
Since we need a function in terms of $t$ it must then be true that $y=1$\\
Thus the final equation is the following
\[
y(t)=1
\]
This equation satisfies the differential equation and initial condition

\section{Problem 4}

\subsection{Part A}

Dividing by $t^2$ gives us
\[
y' + \frac{y}{t} = \frac{1}{t^2}
\]
Thus I will use the linear ODE method with\\
$p(t)=1/t$ and $q(t)=1/(t^2)$\\
The integrating factor is thus as follows
\[
E(t) = exp( ln(t)) = t
\]
Thus our solution is
\[
y = \frac{1}{t} \int \frac{1}{t}
\]
\[
y = \frac{ln(t)}{t}
\]

\subsection{Part B}

Rearranging we have
\[
y' + \frac{y}{t+1} = \frac{ln(t)}{t+1}
\]
I will use the linear ODE method so
\[
p(t) = \frac{1}{t+1}
\]
\[
q(t) = \frac{ln(t)}{t+1}
\]
The integrating factor is thus as follows
\[
E(t) = exp(ln(t+1))
\]
\[
E(t) = t+1
\]
Our solution would thus be
\[
y(t) = \frac{1}{t+1} (\int (t+1)\frac{ln(t)}{t+1}dt + C)
\]
\[
y(t) = \frac{1}{t+1} (\int ln(t)dt + C)
\]
\[
y(t) = \frac{t \cdot ln(t) - t + C}{t+1}
\]

\newpage

\subsection{Part C}

Rearranging this becomes
\[
t y' + y' y = y
\]
Dividing both sides by y we obtain
\[
y'(1 + t/y) = 1
\]
\[
y' = \frac{1}{1+(y/t)^{-1}}
\]
This is a homogeneous ODE. Let 
\[
v(t) = y/t
\]
Writing the ODE in terms of v we obtain
\[
v' t + v = \frac{1}{1+1/v}
\]
This can be written as
\[
v' t + v = \frac{v}{v+1}
\]
Bringing the v over we obtain
\[
v' t = \frac{v - v(v+1)}{v+1}
\]
\[
v' = -\frac{v^2}{v+1} \frac{1}{t}
\]
This is a separable equation so we obtain
\[
\frac{(v+1) dv}{v^2} = - \frac{dt}{t}
\]
\[
(\frac{1}{v} + \frac{1}{v^2})dv = -\frac{dt}{t}
\]
Integrating both sides gives us
\[
ln(v) - \frac{1}{v} = - ln(t) + C
\]
Expanding v we obtain the following
\[
ln(y) - ln(t) - \frac{t}{y} = -ln(t) + C
\]
The $ln(t)$ term cancels. Multiplying $y$ on both sides now yields
\[
y ln(y) - t = Cy
\]
The final implicit form is as follows
\[
y (ln(y) - C) - t = 0
\]

\section{Problem 5}

Here is a sketch of the possible justification of the separation of variables formula\\
We assume the equation has the following form
\[
\frac{dy}{dx} = f(x)g(y)
\]
Assume that $x$ and $y$ are functions of a parameter $t$\\
I am then assuming the following
\[
\frac{dy}{dx} = \frac{dy/dt}{dx/dt}
\]
Let $h(y)=1/g(y)$ and assume prime denotes $dt$ then we can say that
\[
y'(t) h(y(t)) = x'(t) f(x(t))
\]
Let $H$ and $F$ be antiderivatives of $h$ and $f$ respectively\\
Integrating over the same variable $t$ on both sides and applying \\
the rule from integration by substitution it holds that
\[
H(y) = F(x)
\]
This is the same result as what the separation of variables method produces

\end{document}
